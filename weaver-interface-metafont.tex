\font\sixteen=cmbx15
\font\twelve=cmr12
\font\fonteautor=cmbx12
\font\fonteemail=cmtt10
\font\twelvenegit=cmbxti12
\font\twelvebold=cmbx12
\font\trezebold=cmbx13
\font\twelveit=cmsl12
\font\monodoze=cmtt12
\font\it=cmti12
\voffset=0,959994cm % 3,5cm de margem superior e 2,5cm inferior
\parskip=6pt

\def\titulo#1{{\noindent\sixteen\hbox to\hsize{\hfill#1\hfill}}}
\def\autor#1{{\noindent\fonteautor\hbox to\hsize{\hfill#1\hfill}}}
\def\email#1{{\noindent\fonteemail\hbox to\hsize{\hfill#1\hfill}}}
\def\negrito#1{{\twelvebold#1}}
\def\italico#1{{\twelveit#1}}
\def\monoespaco#1{{\monodoze#1}}
\def\iniciocodigo{\lineskip=0pt\parskip=0pt}
\def\fimcodigo{\twelve\parskip=0pt plus 1pt\lineskip=1pt}

\long\def\abstract#1{\parshape 10 0.8cm 13.4cm 0.8cm 13.4cm
0.8cm 13.4cm 0.8cm 13.4cm 0.8cm 13.4cm 0.8cm 13.4cm 0.8cm 13.4cm
0.8cm 13.4cm 0.8cm 13.4cm 0.8cm 13.4cm
\noindent{{\twelvenegit Abstract: }\twelveit #1}}

\def\resumo#1{\parshape  10 0.8cm 13.4cm 0.8cm 13.4cm
0.8cm 13.4cm 0.8cm 13.4cm 0.8cm 13.4cm 0.8cm 13.4cm 0.8cm 13.4cm
0.8cm 13.4cm 0.8cm 13.4cm 0.8cm 13.4cm
\noindent{{\twelvenegit Resumo: }\twelveit #1}}

\def\secao#1{\vskip12pt\noindent{\trezebold#1}\parshape 1 0cm 15cm}
\def\subsecao#1{\vskip12pt\noindent{\twelvebold#1}}
\def\subsubsecao#1{\vskip12pt\noindent{\negrito{#1}}}
\def\referencia#1{\vskip6pt\parshape 5 0cm 15cm 0.5cm 14.5cm 0.5cm 14.5cm
0.5cm 14.5cm 0.5cm 14.5cm {\twelve\noindent#1}}

%@* .

\twelve
\vskip12pt
\titulo{Interface de Usuário Weaver}
\vskip12pt
\autor{Thiago Leucz Astrizi}
\vskip6pt
\email{thiago@@bitbitbit.com.br}
\vskip6pt

\abstract{This article contains an implementation of an interpreter
 for a subset of METAFONT language with some extensions. This subset
 is aimed at allowing the definition of flexible and parameterized
 typographic fonts that could be interpreted and redered on the fly.
 Because of this, we sacrifice some flexibility on the language to be
 able to produce faster results. The implementation will be used by
 Weaver Game Engine as a module and uses modern OpenGL to render the
 fonts.}


\vskip 0.5cm plus 3pt minus 3pt

\resumo{Este artigo contém uma implementação de um interpretador para
 um subconjunto da linguagem METAFONT, com algumas extensões. Este
 subconjunto foi escolhido para permitir a definição de fontes
 tipográficas flexíveis e parametrizadas que podem ser interpretadas
 e renderizadas rapidamente à medida que são lidas. Por causa disso,
 nós sacrificamos alguma flexibilidade da linguagem para sermos
 capazes de criar resultados mais rápidos. A implementação será usada
 pelo Motor de Jogos Weaver como um módulo e usará OpenGL moderno
 para renderizar as fontes.}

\secao{1. Introdução}

METAFONT é uma linguagem feita para descrever fontes tipográficas. Ela
foi criada em 1984 por Donald Knuth e difere de outros formatos de
fontes tipográficas por permitir a criação de fontes à partir da
modificação de parâmetros definidos na descrição básica da
fonte. Desta forma, o projetista de uma fonte não deve criar uma
simples fonte tipográfica, mas uma meta-fonte, a qual pode por sua vez
gerar muitas outras fontes tipográficas diferentes mediante a simples
modificação dos parâmetros.

A especificação original da linguagem METAFONT pode ser encontrada em
[KNUTH, 1989], mas a implementação aqui será apenas de um subconjunto
desta linguagem. Talvez em um futuro, à medida que esta implementação
cresça, as lacunas sejam preenchidas e esta se torne uma implementação
mais completa.

Como este artigo define um subsistema do Motor de Jogos Weaver, e mais
especificamente um subsistema de interface de usuário, nosso objetivo
aqui será definir a seguinte função que irá interpretar um arquivo com
código METAFONT e irá gerar uma interface de usuário com as letrar
lidas como textura:

\iniciocodigo
@<Declaração de Função (metafont.h)@>=
void _Wmetafont_loading(void *(*permanent_alloc)(size_t),
		      void (*permanent_free)(void *),
		      void *(*temporary_alloc)(size_t),
		      void (*temporary_free)(void *),
		      void (*before_loading_interface)(void),
		      void (*after_loading_interface)(void),
		      char *source_filename,
                      struct user_interface *target);
@
\fimcodigo

E nós precisamos inserir o cabeçalho Weaver de interface de usuário:

\iniciociodigo
@<Inclui Cabeçalhos Gerais (metafont.h)@>=
#include "interface.h"
@
\fimcodigo

Mas além de simplesmente carregar as fontes tipográficas lidas para um
arquivo, é importante poder gerar também uma estrutura de dados
representando a fonte lida, para que ela depois possa ser usada por
uma função que renderiza texto:

\iniciocodigo
@<Declaração de Função (metafont.h)@>+=
struct metafont *new_metafont(char *filename);
@
\fimcodigo

\subsecao{1.1. Programação Literária}

Nossa API será escrita usando a técnica de Programação Literária,
proposta por Knuth em [KNUTH, 1984]. Ela consiste em escrever um
programa de computador explicando didaticamente em texto o que se está
fazendo à medida que apresenta o código. Depois, o programa é
compilado através de programas que extraem o código diretamente do
texto didático. O código deve assim ser apresentado da forma que for
mais adequada para a explicação no texto, não como for mais adequado
para o computador.

Seguindo esta técnica, este documento não é uma simples documentação
do nosso código. Ele é por si só o código. A parte que será extraída e
compilada posteriormente pode ser identificada como sendo o código
presente em fundo cinza. Geralmente começamos cada trecho de código
com um título que a nomeia. Por exemplo, imediatamente antes desta
subseção nós apresentamos uma série de declarações. E como pode-se
deduzir pelo título delas, a maioria será posteriormente posicionada
dentro de um arquivo chamado \monoespaco{metafont.h}.

Podemos apresentar aqui a estrutura do arquivo
\monoespaco{metafont.h}:

\iniciocodigo
@(src/metafont.h@>=
#ifndef __WEAVER_METAFONT
#define __WEAVER_METAFONT
#ifdef __cplusplus
extern "C" {
#endif
#include <stdbool.h> // Define tipo 'bool'
#if !defined(_WIN32)
#include <sys/param.h> // Necessário no BSD, mas causa problema no Windows
#endif
@<Inclui Cabeçalhos Gerais (metafont.h)@>
//@<Macros Gerais (metafont.h)@>
@<Estrutura de Dados (metafont.h)@>
@<Declaração de Função (metafont.h)@>
#ifdef __cplusplus
}
#endif
#endif
/@
\fimcodigo

O código acima mostra a burocracia padrão para definir um cabeçalho
para nossa API em C. As duas primeiras linhas mais a última são macros
que garantem que esse cabeçalho não será inserido mais de uma vez em
uma mesma unidade de compilação. As linhas 3, 4, 5, assim como a
penúltima, antepenúltima e a antes da antepenúltima tornam o cabeçalho
adequado a ser inserido em código C++. Essas linhas apenas avisam que
o que definirmos ali deve ser encarado como código C. Por isso o
compilador está livre para fazer otimizações sabendo que não usaremos
recursos da linguagem C++, como sobrecarga de operadores. Logo em
seguida, inserimos um cabeçalho que nos permite declarar o tipo
booleano. E tem também uma parte em vermelha. Note que uma delas é
``Declaração de Função (metafont.h)'', o mesmo nome apresentado no
trecho de código mostrado quando descrevemos nossa API antes dessa
subseção. Isso significa que aquele código visto antes será depois
inserido ali. As outras partes em vermelho representam código que
ainda iremos definir nas seções seguintes.

Caso queira observar o que irá no arquivo \monoespaco{metafont.c}
associado a este cabeçaho, o código será este:

\iniciocodigo
@(src/metafont.c@>=
#include "metafont.h"
@<Cabeçalhos Locais (metafont.c)@>
//@<Macros Locais (metafont.c)@>
@<Estrutura de Dados Locais (metafont.c)@>
//@<Variáveis Locais (metafont.c)@>
@<Declaração de Função Local (metafont.c)@>
@<Funções Auxiliares Locais (metafont.c)@>
//@<Definição de Funções da API (metafont.c)@>
@
\fimcodigo

Todo o código que definiremos e explicaremos a seguir será posicionado
nestes dois arquivos. Além deles, nenhum outro arquivo será criado.

\secao{2. Analisador Léxico}

A primeira coisa a criar para uma linguagem é seu analisador
léxico. Ele irá ler o código-fonte da linguagem presente em um arquivo
e irágerar uma lista de ``tokens'', sendo cada token a unidade mais
básica da linguagem, basicamente uma palavra.

METAFONT reconhece três tipos de tokens: numéricos, strings e
simbólicos. Na prática nós iremos dividir os tokens simbólicos em
diferentes tipos de subtokens para podermos armazená-los e lê-los de
maneira mais eficiente.

Um token numérico será representado internamente como um número em
ponto flutuante. Nisso iremos diferir do METAFONT original que usava
uma representação própria de números. Em nome de operações mais
rápidas, usaremos a forma moderna de representar números. Então essa é
a aparência de um token numérico:

\iniciocodigo
@<Estrutura de Dados Locais (metafont.c)@>=
#define TYPE_NUMERIC 1
struct numeric_token{
  int type;   // Deve ser igual a 'TYPE_NUMERIC'
  void *next;
#if defined(W_DEBUG_METAFONT)
  int line;
#endif
  float value;
};
@
\fimcodigo

E isso é como representaremos tokens de strings:

\iniciocodigo
@<Estrutura de Dados Locais (metafont.c)@>+=
#define TYPE_STRING 2
struct string_token{
  int type;   // Deve ser igual a 'TYPE_STRING'
  void *next;
#if defined(W_DEBUG_METAFONT)
  int line;
#endif
  char value[5];
};
@
\fimcodigo

Nós iremos armazenar apenas os primeiros 5 bytes de cada string que
encontrarmos, mesmo que no código-fonte a string seja maior. Isso será
feito porque ao contrário do METAFONT original, a única utilidade para
strings aqui será indicar qual caractere Unicode deve estar associado
a cada glifo definido. Para isso nós precisamos de apenas 5 bytes (no
máximo 4 para o caractere mais o byte 0 final). E também podemos usar
eles para associar certos glifos a ligaduras como ``ff'', que também
cabe em 5 bytes.

Já no caso de tokens simbólicos, precisamos armazenar todo o seu
conteúdo para podermos saber qual token temos, e vamos alocar
dinamicamente de acordo com o tamanho do símbolo:

\iniciocodigo
@<Estrutura de Dados Locais (metafont.c)@>+=
#define TYPE_SYMBOLIC 3
struct symbolic_token{
  int type;   // Deve ser igual a 'TYPE_SYMBOLIC'
  void *next;
#if defined(W_DEBUG_METAFONT)
  int line;
#endif
  char *value;
};
@
\fimcodigo

Mas isso pode consumir mais espaço na memória do que é
necessário. Alguns símbolos são muito comuns e são definidos por
padrão na própria linguagem e podem usar menos espaço que isso. Por
exemplo, parênteses e vírgulas. Podemos representar eles com a
seguinte estrutura:

\iniciocodigo
@<Estrutura de Dados Locais (metafont.c)@>+=
#define TYPE_OPEN_PARENTHESIS  4 // '('
#define TYPE_CLOSE_PARENTHESIS 5 // ')'
#define TYPE_COMMA             6 // ','
#define TYPE_SEMICOLON         7 // ';'
@<METAFONT: Definição de Token Simbólico@>
// (...)
struct generic_token{
  int type;   // Should be one of the above
  void *next;
#if defined(W_DEBUG_METAFONT)
  int line;
#endif
};
@
\fimcodigo

Podemos definir posteriormente muitos outros tipos de tokens
simbólicos reservados. Basta definirmos eles associando a um número
único maior do que 6. Qualquer token cujo tipo é um número igual ou
maior que 3 é simbólico.

Todo token possui um ponteiro para um próximo token. Isso ocorre
porque eles geralmente formarão uma lista encadeada de tokens. Para
desalocar a memória ocupada por uma lista de tokens dada uma função de
desalocação e apagá-los, pode-se usar a seguinte função:

\iniciocodigo
@<Funções Auxiliares Locais (metafont.c)@>=
void free_token_list(void (*dealloc)(void *), void *token_list){
  struct generic_token *p, *p_next;
  if(dealloc == NULL || token_list == NULL)
    return;
  p = token_list;
  while(p != NULL){
    p_next = p -> next;
    if(p -> type == TYPE_SYMBOLIC)
      dealloc(((struct symbolic_token *) p) -> value);
    dealloc(p);
    p = p_next;
  }
}
@
\fimcodigo

Agora a função que fará a análise léxica. Ela receberá como argumento
uma função de alocação de memória e uma string com caminho para
arquivo com código-fonte METAFONT. Ela retornará uma lista encadeada
de tokens:

\iniciocodigo
@<Funções Auxiliares Locais (metafont.c)@>=
void *lexer(void *(*alloc)(size_t), void (*dealloc)(void *), char *path){
  FILE *fp;
  char c;
  int line = 1;
  void *first_token = NULL, *last_token = NULL;
  fp = fopen(path, "r");
  if(fp == NULL)
    return NULL;
  while((c = fgetc(fp)) != EOF){
    char next_char = fgetc(fp);
    ungetc(next_char, fp);
    if(c == '\n'){
      line ++;
      continue;
    }
    @<Análise Léxica: Regra 1@>
    @<Análise Léxica: Regra 2@>
    @<Análise Léxica: Regra 3@>
    @<Análise Léxica: Regra 4@>
    @<Análise Léxica: Regra 5@>
    @<Análise Léxica: Regra 6@>
    // Nenhuma regra aplicada: erro
    fprintf(stderr, "ERROR: %s:%d: Unknown character: '%c'\n",
            path, line, c);
    free_token_list(dealloc, first_token);
    return NULL;
  }
  return first_token;
}
@
\fimcodigo

Oh, como usamos o tipo \monoespaco{FILE}, precisamos inserir o
cabeçalho de entrada e saída padrão para podermos abrir e ler
arquivos:

\iniciocodigo
@<Cabeçalhos Locais (metafont.c)@>=
#include <stdio.h>
@
\fimcodigo

As regras para ler os tokens consiste em ler cada linha do código
fonte aplicando as seguintes regras para cada novo caractere lido:

1) Se o próximo caractere é um espaço ou um ponto que não é sucedido
por um dígito decimal ou um novo ponto, ignore-o e siga em frente.

\iniciocodigo
@<Análise Léxica: Regra 1@>=
if(c == ' ' || (c == '.' && next_char != '.' && !isdigit(next_char)))
  continue;
@
\fimcodigo

Como estamos usando a função \monoespaco{isdigit}, devemos inserir o
seguinte cabeçalho que a declara:

\iniciocodigo
@<Cabeçalhos Locais (metafont.c)@>=
#include <ctype.h>
@
\fimcodigo


2) Se o próximo caractere é o símbolo de porcentagem, ignore-o e
ignore todos os outros caracteres seguintes na linha. Este símbolo
marca o começo de comentários na linguagem.

\iniciocodigo
@<Análise Léxica: Regra 2@>=
if(c == '%'){
  do{
    c = fgetc(fp);
  } while(c != '\n' && c != EOF);
  ungetc(c, fp);
  continue;
}
@
\fimcodigo

3) Se o próximo caractere é um dígito decimal ou um ponto seguido por
um dígito decimal, então o próximo token será numérico. E é formado
pela maior sequência possível de ser lida envolvendo dígitos decimais
e opcionalmente um único ponto representando o ponto decimal.

\iniciocodigo
@<Análise Léxica: Regra 3@>=
if((c == '.' && isdigit(next_char)) || isdigit(c)){
  char buffer[256];
  struct numeric_token *new_token = (struct numeric_token *)
                                      alloc(sizeof(struct numeric_token));
  if(new_token == NULL){
    free_token_list(dealloc, first_token);
    return NULL;
  }
  new_token -> type = TYPE_NUMERIC;
  new_token -> next = NULL;
#if defined(W_DEBUG_METAFONT)
  new_token -> line = line;
#endif
  int i = 0;
  int number_of_dots = (c == '.');
  buffer[i] = c;
  i ++;
  do{
    c = fgetc(fp);
    if(c == '.')
      number_of_dots ++;
    buffer[i] = c;
    i ++;
  } while(isdigit(c) || (c == '.' && number_of_dots == 1));
  ungetc(c, fp);
  i --;
  buffer[i] = '\0';
  new_token -> value = atof(buffer);
  if(first_token == NULL)
    first_token = last_token = new_token;
  else{
    ((struct generic_token *) last_token) -> next = new_token;
    last_token = new_token;
  }
  continue;
}
@
\fimcodigo


4) Se o próximo caractere é uma aspa dupla, então o próximo token deve
ser uma string e será composto pela sequência de caracteres até a
próxima aspa dupla, que deve estar na mesma linha. Se existir uma aspa
dupla iniciando uma string, mas não existir outra para finalizar a
string na mesma linha, um erro será gerado.

\iniciocodigo
@<Análise Léxica: Regra 4@>=
if(c == '"'){
  struct string_token *new_token = (struct string_token *)
                                     alloc(sizeof(struct string_token));
  if(new_token == NULL){
    free_token_list(dealloc, first_token);
    return NULL;
  }
  new_token -> type = TYPE_STRING;
  new_token -> next = NULL;
#if defined(W_DEBUG_METAFONT)
  new_token -> line = line;
#endif
  int i = 0;
  do{
    c = fgetc(fp);
    if(i < 5){
      new_token -> value[i] = c;
      i ++;
    }
  } while(c != '"' && c != '\n' && c != EOF);
  i --;
  new_token -> value[i] = '\0';
  if(c == '\n' || c == EOF){
    fprintf(stderr, "ERROR: %s:%d: Incomplete String.\n", path, line);
    dealloc(new_token);
    free_token_list(dealloc, first_token);
    return NULL;
  }
  if(first_token == NULL)
    first_token = last_token = new_token;
  else{
    ((struct generic_token *) last_token) -> next = new_token;
    last_token = new_token;
  }
  continue;
}
@
\fimcodigo

5) Se o próximo caractere for um parêntese, ponto-e-vírgula ou uma
vírgula, então o próximo token será um token simbólico formado por
este único caractere.

\iniciocodigo
@<Análise Léxica: Regra 5@>=
if(c == '(' || c == ')' || c == ',' || c == ';'){
  struct generic_token *new_token =
     (struct generic_token *) alloc(sizeof(struct generic_token));
  if(new_token == NULL){
    free_token_list(dealloc, first_token);
    return NULL;
  }
  if(c == '(')
    new_token -> type = TYPE_OPEN_PARENTHESIS;
  else if(c == ')')
    new_token -> type = TYPE_CLOSE_PARENTHESIS;
  else if(c == ';')
    new_token -> type = TYPE_SEMICOLON;
  else
    new_token -> type = TYPE_COMMA;
  new_token -> next = NULL;
#if defined(W_DEBUG_METAFONT)
  new_token -> line = line;
#endif
  if(first_token == NULL)
    first_token = last_token = new_token;
  else{
    ((struct generic_token *) last_token) -> next = new_token;
    last_token = new_token;
  }
  continue;
}
@
\fimcodigo

6) Caso contrário, o próximo token será simbólico e formado pela
sequência mais longa dentre 12 famílias de caracteres:

\iniciocodigo
@<Análise Léxica: Regra 6@>=
{
  char buffer[256];
  int i = 0;
  buffer[0] = '\0';
  // Buffer é lido de acordo com as subregras 6-a a 6-l
  @<Análise Léxica: Regra A@>
  @<Análise Léxica: Regra B@>
  @<Análise Léxica: Regra C@>
  @<Análise Léxica: Regra D@>
  @<Análise Léxica: Regra E@>
  @<Análise Léxica: Regra F@>
  @<Análise Léxica: Regra G@>
  @<Análise Léxica: Regra H@>
  @<Análise Léxica: Regra I@>
  @<Análise Léxica: Regra J@>
  @<Análise Léxica: Regra K@>
  @<Análise Léxica: Regra L@>
  // De acordo com conteúdo do buffer, gera o próximo token
  @<Análise Léxica: Gera Token Simbólico Reservado@>
  @<Análise Léxica: Gera Token Simbólico Genérico@>
}
@
\fimcodigo


a) A primeira família de letras são as letras de A a Z maiúsculas e
minúsculas além do ``underline''.

\iniciocodigo
@<Análise Léxica: Regra A@>=
if(isalpha(c) || c == '_'){
  do{
    buffer[i] = c;
    i ++;
    c = fgetc(fp);
  } while(isalpha(c) || c == '_');
  ungetc(c, fp);
  buffer[i] = '\0';
}
@
\fimcodigo

b) A segunda família são os símbolos de maior, menor, igual, dois
pontos e ``|''.

\iniciocodigo
@<Análise Léxica: Regra B@>=
else if(c == '>' || c == '<' || c == '=' || c == ':' || c == '|'){
  do{
    buffer[i] = c;
    i ++;
    c = fgetc(fp);
  } while(c == '>' || c == '<' || c == '=' || c == ':' || c == '|');
  ungetc(c, fp);
  buffer[i] = '\0';
}
@
\fimcodigo

c) Acento agudo e grave.

\iniciocodigo
@<Análise Léxica: Regra C@>=
else if(c == '`' || c == '\''){
  do{
    buffer[i] = c;
    i ++;
    c = fgetc(fp);
  } while(c == '`' || c == '\'');
  ungetc(c, fp);
  buffer[i] = '\0';
}
@
\fimcodigo

d) Mais e menos.

\iniciocodigo
@<Análise Léxica: Regra D@>=
else if(c == '+' || c == '-'){
  do{
    buffer[i] = c;
    i ++;
    c = fgetc(fp);
  } while(c == '+' || c == '-');
  ungetc(c, fp);
  buffer[i] = '\0';
}
@
\fimcodigo

e) Barra, barra invertida e símbolo de multiplicação.

\iniciocodigo
@<Análise Léxica: Regra E@>=
else if(c == '\\' || c == '/' || c == '*'){
  do{
    buffer[i] = c;
    i ++;
    c = fgetc(fp);
  } while(c == '\\' || c == '/' || c == '*');
  ungetc(c, fp);
  buffer[i] = '\0';
}
@
\fimcodigo

f) Ponto de interrogação e exclamação.

\iniciocodigo
@<Análise Léxica: Regra F@>=
else if(c == '?' || c == '!'){
  do{
    buffer[i] = c;
    i ++;
    c = fgetc(fp);
  } while(c == '?' || c == '!');
  ungetc(c, fp);
  buffer[i] = '\0';
}
@
\fimcodigo

g) Cerquilha, ``e'' comercial, arroba e cifrão.

\iniciocodigo
@<Análise Léxica: Regra G@>=
else if(c == '#' || c == '&' || c == '@@' || c == '$'){
  do{
    buffer[i] = c;
    i ++;
    c = fgetc(fp);
  } while(c == '#' || c == '&' || c == '@@' || c == '$');
  ungetc(c, fp);
  buffer[i] = '\0';
}
@
\fimcodigo

h) Acento circunflexo e til.

\iniciocodigo
@<Análise Léxica: Regra H@>=
else if(c == '^' || c == '~'){
  do{
    buffer[i] = c;
    i ++;
    c = fgetc(fp);
  } while(c == '^' || c == '~');
  ungetc(c, fp);
  buffer[i] = '\0';
}
@
\fimcodigo

i) Abrir colchetes.

\iniciocodigo
@<Análise Léxica: Regra I@>=
else if(c == '['){
  do{
    buffer[i] = c;
    i ++;
    c = fgetc(fp);
  } while(c == '[');
  ungetc(c, fp);
  buffer[i] = '\0';
}
@
\fimcodigo

j) Fechar colchetes.

\iniciocodigo
@<Análise Léxica: Regra J@>=
else if(c == ']'){
  do{
    buffer[i] = c;
    i ++;
    c = fgetc(fp);
  } while(c == ']');
  ungetc(c, fp);
  buffer[i] = '\0';
}
@
\fimcodigo

k) Abrir e fechar chaves.

\iniciocodigo
@<Análise Léxica: Regra K@>=
else if(c == '{' || c == '}'){
  do{
    buffer[i] = c;
    i ++;
    c = fgetc(fp);
  } while(c == '{' || c == '}');
  ungetc(c, fp);
  buffer[i] = '\0';
}
@
\fimcodigo

l) Ponto.

\iniciocodigo
@<Análise Léxica: Regra L@>=
else if(c == '.'){
  do{
    buffer[i] = c;
    i ++;
    c = fgetc(fp);
  } while(c == '.');
  ungetc(c, fp);
  buffer[i] = '\0';
}
@
\fimcodigo


Depois de ler todos os caracteres do próximo token, primeiro vemos se
estamos diante de um token simbólico reservado. Estes são tokens
simbólicos com o que seriam palavras-chave da linguagem. Por hora
ainda não iremos mostrá-los, mas testar se é o caso é apenas uma
questão de usar a função \monoespaco{strcmp}. Vamos inclusive inserir
o cabeçalho para poder comparar e manipular strings:

\iniciocodigo
@<Cabeçalhos Locais (metafont.c)@>=
#include <string.h>
@
\fimcodigo

Se não for um token simbólico reservado, e se existir algo copiado
para o nosso buffer, então geramos um novo token simbólico genérico:

\iniciocodigo
@<Análise Léxica: Gera Token Simbólico Genérico@>=
if(buffer[0] != '\0'){
  buffer[255] = '\0';
  size_t buffer_size = strlen(buffer) + 1;
  struct symbolic_token *new_token =
     (struct symbolic_token *) alloc(sizeof(struct symbolic_token));
  if(new_token == NULL){
    free_token_list(dealloc, first_token);
    return NULL;
  }
  new_token -> type = TYPE_SYMBOLIC;
  new_token -> next = NULL;
#if defined(W_DEBUG_METAFONT)
  new_token -> line = line;
#endif
  new_token -> value = (char *) alloc(buffer_size);
  memcpy(new_token -> value, buffer, buffer_size);
  if(first_token == NULL)
    first_token = last_token = new_token;
  else{
    ((struct generic_token *) last_token) -> next = new_token;
    last_token = new_token;
  }
  continue;
}
@
\fimcodigo


\secao{3. Programas METAFONT}

Quando avaliamos um programa METAFONT, vamos precisar de duas
extruturas adicionais. A primeira delas, a qual chamaremos
de \monoespaco{struct metafont} irá conter as informações finais da
fonte tipográfica que foi lida e que terá tudo que for necessário para
que tal fonte possa ser usada para renderizar cada um dos glifos. A
segunda estrutura, a qual chamaremos de \monoespaco{struct context}
representa o estado atual do analisador sintático e representa
informações que devemos saber para poder continuar interpretando
corretamente uma lista de tokens. Essa segunda estrutura pode ser
descartada depois de terminarmos de ler os tokens de nossa fonte.

A primeira coisa que nosso analisador sintático precisa saber é que
todo programa METAFONT é uma sequência de declarações (possivelmente
vazia) finalizada por um token simbólico que
representa \monoespaco{end} ou \monoespaco{dump}:

\alinhaverbatim
<Programa> -> <Lista de Declarações> end | <Lista de Declarações> dump
\alinhanormal

Para nós não haverá nenhuma diferença entre \monoespaco{end}
e \monoespaco{dump}. No METAFONT original, o primeiro era usado para
finalizar o programa de uma fonte tipográfica e o segundo para
finalizar a descrição de um arquivo-base (algo como uma biblioteca
padrão básica).

Vamos definir então um novo tipo de token simbólico que será uma
abreviação deste símbolo de finalização, seja qual ele for:

\iniciocodigo
@<METAFONT: Definição de Token Simbólico@>=
#define TYPE_END             8 // O token simbólico 'end' ou 'dump'
@
\fimcodigo

E geraremos esse tipo de token no nosso analisador léxico quando
encontrarmos uma das duas palavras-chaves (tecnicamente METAFONT chama
elas não de palavras-chaves, mas de ``sparks''):

\iniciocodigo
@<Análise Léxica: Gera Token Simbólico Reservado@>=
if(!strcmp(buffer, "end") || !strcmp(buffer, "dump")){
  struct generic_token *new_token =
     (struct generic_token *) alloc(sizeof(struct generic_token));
  if(new_token == NULL){
    free_token_list(dealloc, first_token);
    return NULL;
  }
  new_token -> type = TYPE_END;
  new_token -> next = NULL;
#if defined(W_DEBUG_METAFONT)
  new_token -> line = line;
#endif
  if(first_token == NULL)
    first_token = last_token = new_token;
  else{
    ((struct generic_token *) last_token) -> next = new_token;
    last_token = new_token;
  }
  continue;
}
@
\fimcodigo

A primeira função de análise sintática que definiremos é a que
reconhece um programa inteiro. Ela basicamente checa se o programa é
corretamente finalizado e se não o for, retorna erro. Caso ele esteja
corretamente finalizado, ela passa o programa para a próxima função de
análise sintática que interpreta lista de declarações, marcando o
começo e o fim da lista de declarações:

\iniciocodigo
@<Funções Auxiliares Locais (metafont.c)@>=
bool eval_program(struct metafont *mf, struct context *cx,
                  void *token_list){
  struct generic_token *end = (struct generic_token *) token_list;
  struct generic_token *previous = NULL;
  while(end != NULL && end -> type != TYPE_END){
    previous = end;
    end = (struct generic_token *) (end -> next);
  }
  if(end == NULL){
#if defined(W_DEBUG_METAFONT)
    fprintf(stderr, "METAFONT: Error: %s:%d: Program not finished with "
                    "'end' or 'dump'.\n", mf -> file,
                    (previous == NULL)?(1):(previous -> line));
#endif
    return false;
  }
  if(end == token_list)
    return true;
  return eval_list_of_statements(mf, cx, token_list, previous);
}
@
\fimcodigo

A estrutura de contexto ainda não será definida inteiramente. Iremos
mostrar o conteúdo dela à medida que for necessário:

\iniciocodigo
@<Estrutura de Dados (metafont.h)@>=
struct context{
  //@<Atributos (struct context)@>
};
@
\fimcodigo

E como indicado acima, um dos conteúdos de \monoespaco{struct
metafont} é o arquivo de onde saiu a fonte que ele representa:

\iniciocodigo
@<Estrutura de Dados (metafont.h)@>+=
struct metafont{
  char *file;
  //@<Atributos (struct metafont)@>
};
@
\fimcodigo

\secao{4. Listas de Declarações}

Uma lista de declarações é uma série de declarações (potencialmente
vazias) sempre separadas por ponto-e-vírgula:

\alinhaverbatim
<Lista de Declarações> -> <Vazio> | <Declaração> ; <Lista de Declarações>
\alinhanormal

A função que iremos usar para avaliar e interpretar uma lista de
declarações é:

\iniciocodigo
@<Declaração de Função Local (metafont.c)@>=
bool eval_list_of_statements(struct metafont *mf, struct context *cx,
                            void *begin_token_list, void *end_token_list);
@
\fimcodigo

O que esta função fará será iterar sobre cada declaração e executar em
cada uma delas na ordem, a função de avaliação de declaração
individual. Para isso seu trabalho será ver onde cada declaração é
delimitada por cada ponto-e-vírgula e passar os limites de cada
declaração para essa próxima função de avaliação:

\iniciocodigo
@<Funções Auxiliares Locais (metafont.c)@>=
bool eval_list_of_statements(struct metafont *mf, struct context *cx,
                            void *begin_token_list, void *end_token_list){
  bool ret = true;
  struct generic_token *begin, *end = NULL;
  begin = (struct generic_token *) begin_token_list;
  while(begin != NULL){
    while(begin != NULL && begin -> type == TYPE_SEMICOLON){
      if(begin != end_token_list)
        begin = (struct generic_token *) begin -> next;
      else
        begin = NULL;
    }
    end = begin;
    if(end != NULL){
      while(end != end_token_list &&
            ((struct generic_token *) (end -> next)) -> type !=
              TYPE_SEMICOLON)
        end = end -> next;
    }
    if(begin != NULL){
      // eval_statement(mf, cx, begin, end);
      begin = end -> next;
    }
  }
  return ret;
}
@
\fimcodigo

A função acima itera sobre cada declaração, passando direto por
declarações vazias, e para cada declaração não-vazia, faz com que os
ponteiros \monoespaco{begin} e \monoespaco{end} a delimitem. Durante a
iteração avançamos os ponteiros para os próximos tokens, mas cuidamos
para nunca sair de dentro da região delimitada
por \monoespaco{begin\_token\_list} e \monoespaco{end\_token\_list}.

\secao{5. Declaração Composta}

Uma declaração individual na linguagem pode na verdade ser composta
por muitas outras declarações. A sintaxe para isso é:

\alinhaverbatim
<Declaração> -> <Vazia> | <Composta> | <Título> | <Equação> |
                <Atribuição> | <Declaração> | <Definição> | <Comando>
<Composta> -> begingroup <Lista de Declarações> <Não-Título> endgroup
\alinhanormal

Já sabemos o que é uma lista de declarações e já temos uma função que
as avalia. Já a declaração não-título, por enquanto iremos tratar como
sendo qualquer tipo de declaração. Pois nossa versão da linguagem
ainda não suporta declarações de título (seriam strings
isoladas). Desta forma, para nós, uma declaração composta é apenas uma
lista de declarações delimitada por tokens \monoespaco{begingroup}
e \monoespaco{endgroup}.

Por isso vamos ter que levar em conta a existência destes dois tipos
novos de tokens simbólicos no nosso analisador léxico:

\iniciocodigo
@<METAFONT: Definição de Token Simbólico@>+=
#define TYPE_BEGINGROUP             9 // O token simbólico 'begingroup'
#define TYPE_ENDGROUP              10 // O token simbólico 'endgroup'
@
\fimcodigo

E adicionamos a descrição de como gerá-los na análise léxica:

\iniciocodigo
@<Análise Léxica: Gera Token Simbólico Reservado@>+=
else if(!strcmp(buffer, "begingroup") || !strcmp(buffer, "endgroup")){
  struct generic_token *new_token =
     (struct generic_token *) alloc(sizeof(struct generic_token));
  if(new_token == NULL){
    free_token_list(dealloc, first_token);
    return NULL;
  }
  if(buffer[0] == 'b')
    new_token -> type = TYPE_BEGINGROUP;
  else
    new_token -> type = TYPE_ENDGROUP;
  new_token -> next = NULL;
#if defined(W_DEBUG_METAFONT)
  new_token -> line = line;
#endif
  if(first_token == NULL)
    first_token = last_token = new_token;
  else{
    ((struct generic_token *) last_token) -> next = new_token;
    last_token = new_token;
  }
  continue;
}
@
\fimcodigo



%%%%%%%%%%%%%%%%%%%%%%%%%%%%%%%%%%%%%

\secao{Referências}

\referencia{Knuth, D. E. (1984) ``Literate Programming'', The Computer
  Journal, Volume 27, Edição 2, Páginas 97--111.}

\referencia{Knuth, D. E. (1989) ``The METAFONT book'', Addison-Wesley
 Longsman Publishing Co., Inc}

\fim

\input tex/epsf.tex
\font\sixteen=cmbx15
\font\twelve=cmr12
\font\fonteautor=cmbx12
\font\fonteemail=cmtt10
\font\twelvenegit=cmbxti12
\font\twelvebold=cmbx12
\font\trezebold=cmbx13
\font\twelveit=cmsl12
\font\monodoze=cmtt12
\font\it=cmti12
\voffset=0,959994cm % 3,5cm de margem superior e 2,5cm inferior
\parskip=6pt

\def\titulo#1{{\noindent\sixteen\hbox to\hsize{\hfill#1\hfill}}}
\def\autor#1{{\noindent\fonteautor\hbox to\hsize{\hfill#1\hfill}}}
\def\email#1{{\noindent\fonteemail\hbox to\hsize{\hfill#1\hfill}}}
\def\negrito#1{{\twelvebold#1}}
\def\italico#1{{\twelveit#1}}
\def\monoespaco#1{{\monodoze#1}}
\def\iniciocodigo{\lineskip=0pt\parskip=0pt}
\def\fimcodigo{\twelve\parskip=0pt plus 1pt\lineskip=1pt}

\long\def\abstract#1{\parshape 10 0.8cm 13.4cm 0.8cm 13.4cm
0.8cm 13.4cm 0.8cm 13.4cm 0.8cm 13.4cm 0.8cm 13.4cm 0.8cm 13.4cm
0.8cm 13.4cm 0.8cm 13.4cm 0.8cm 13.4cm
\noindent{{\twelvenegit Abstract: }\twelveit #1}}

\def\resumo#1{\parshape  10 0.8cm 13.4cm 0.8cm 13.4cm
0.8cm 13.4cm 0.8cm 13.4cm 0.8cm 13.4cm 0.8cm 13.4cm 0.8cm 13.4cm
0.8cm 13.4cm 0.8cm 13.4cm 0.8cm 13.4cm
\noindent{{\twelvenegit Resumo: }\twelveit #1}}

\def\secao#1{\vskip12pt\noindent{\trezebold#1}\parshape 1 0cm 15cm}
\def\subsecao#1{\vskip12pt\noindent{\twelvebold#1}}
\def\subsubsecao#1{\vskip12pt\noindent{\negrito{#1}}}
\def\referencia#1{\vskip6pt\parshape 5 0cm 15cm 0.5cm 14.5cm 0.5cm 14.5cm
0.5cm 14.5cm 0.5cm 14.5cm {\twelve\noindent#1}}

%@* .

\twelve
\vskip12pt
\titulo{Weaver User Interface}
\vskip12pt
\autor{Thiago Leucz Astrizi}
\vskip6pt
\email{thiago@@bitbitbit.com.br}
\vskip6pt

\abstract{This article contains an implementation of an interpreter
 for a subset of METAFONT language with some extensions. This subset
 is aimed at allowing the definition of flexible and parameterized
 typographic fonts that could be interpreted and redered on the fly.
 Because of this, we sacrifice some flexibility on the language to be
 able to produce faster results. The implementation will be used by
 Weaver Game Engine as a module and uses modern OpenGL to render the
 fonts.}

\secao{1. Introduction}

METAFONT is a language made to describe typographical fonts. It was
created on 1984 by Donald Knuth and differs from other formats for
allowing a designer to create different fonts merely by changing basic
parameters in the base description of the font. This way, a designer
should not create a single typographic font, but a meta-font from
which new fonts could be obtained changing these basic parameters.

The original specification for the METAFONT language can be found in
[KNUTH, 1989], but the implementation described here will be just a
subset of that language. Perhaps in the future we fill the gaps and
this implementation becames more complete and compatible with the full
specification.

As this article defines a subsystem for Weaver Game Engine, and more
specifically a subsystem for the user interface modulus, our objective
here will be define the following function that will interpret a file
with METAFONT source code and load as user interface using the letters
read as texture:

\iniciocodigo
@<Function Declaration (metafont.h)@>=
void _Wmetafont_loading(void *(*permanent_alloc)(size_t),
		      void (*permanent_free)(void *),
		      void *(*temporary_alloc)(size_t),
		      void (*temporary_free)(void *),
		      void (*before_loading_interface)(void),
		      void (*after_loading_interface)(void),
		      char *source_filename,
                      struct user_interface *target);
@
\fimcodigo

And we will need the header for Weaver user interfaces:

\iniciociodigo
@<Include General Headers (metafont.h)@>=
#include "interface.h"
@
\fimcodigo

But besides loading the typographical fonts to a texture, we also want
to create a font structure that could be used to render text for other
functions:

\iniciocodigo
@<Function Declaration (metafont.h)@>+=
struct metafont *new_metafont(char *filename);
@
\fimcodigo

\subsecao{1.1. Literate Programming}

Our API will be written using the literate programming technique,
proposed by Knuth on [Knuth, 1984]. It consist in writting a computer
program explaining didactically in a text what is being done while
presenting the code. The program is compiled extracting the computer
code directly from the didactical text. The code shall be presented in
a way and order such that it is best for explaining for a human. Not
how it would be easier to compile.

Using this technique, this document is not a simple documentation for
our code. It is the code by itself. The part that will be extracted to
be compiled can be identified by a gray background. We begin each
piece of code by a title that names it. For example, immediately
before this subsection we presented a series of function
declarations. And how one could deduct by the title, most of them will
be positioned in the file \monoespaco{metafont.h}.

We can show the structure of the file \monoespaco{metafont.h}:

\iniciocodigo
@(src/metafont.h@>=
#ifndef __WEAVER_METAFONT
#define __WEAVER_METAFONT
#ifdef __cplusplus
extern "C" {
#endif
#include <stdbool.h> // Define  'bool' type
#if !defined(_WIN32)
#include <sys/param.h> // Needed on BSD, but does not exist on Windows
#endif  
@<Include General Headers (metafont.h)@>
//@<General Macros (metafont.h)@>
@<Data Structures (metafont.h)@>
@<Function Declaration (metafont.h)@>
#ifdef __cplusplus
}
#endif
#endif
@
\fimcodigo

The cde above shows the default boureaucracy to define a header for
our C API. The two first lines and the last one are macros that ensure
that this header will not be inserted more than once in a single
compiling unit. The lines 3, 4, 5 and the three lines before the last
one make the header adequate to be used in C++ code. This tells the
compiler that we are using C code and that therefore, the compiler is
free to use optimizations assuming that we will not use C++ exclusive
techniques, like operator overloading. Next we include a header that
will let us to use boolean variables. And there are some parts in
red. Note that one of them is called ``Function Declaration
(weaver.h)'', the same title used in most of the code declared
previously. This means that all the previous code with this title will
be inserted in that position inside this file. The other parts in red
represent code that we will define in the next sections.

If you want to know how is the \monoespaco{metafont.c} file related
with this header, its structure is:

\iniciocodigo
@(src/metafont.c@>=
#include "metafont.h"
@<Local Headers (metafont.c)@>
//@<Local Macros (metafont.c)@>
@<Local Data Structures (metafont.c)@>
//@<Local Variables (metafont.c)@>
@<Local Function Declaration (metafont.c)@>
@<Auxiliary Local Functions (metafont.c)@>
//@<API Functions Definition (metafont.c)@>
@
\fimcodigo

All the code presented in this document will be placed in one of these
two files. Besides them, no other file will be created.

\secao{2. Lexer}

The first thing to be created for a language is its lexer. It will
read the source code in a file and will output a list of tokens, where
tokens are the most basic unit in the language. A token is like a
word.

METAFONT recognizes three kind of tokens: numeric, strings and
symbolic. In practice we will subdivide the symbolic tokens in several
kind of subtokens to store and read them in a more efficient way.

A numeric token will be represented internally as a floating point
number. This is different than specified in the original METAFONT,
where a custom numeric representation was used. As here we are
interested in more speed, we will choose floating point numbers
because they have nowadays more hardware support. This is how a
numeric token is represented:

\iniciocodigo
@<Local Data Structures (metafont.c)@>=
#define TYPE_NUMERIC 1
struct numeric_token{
  int type;   // Should be equal 'TYPE_NUMERIC'
  void *next;
  float value;
};
@
\fimcodigo

And this is how we will represent string tokens:

\iniciocodigo
@<Local Data Structures (metafont.c)@>+=
#define TYPE_STRING 2
struct string_token{
  int type;   // Should be equal 'TYPE_STRING'
  void *next;
  char value[5];
};
@
\fimcodigo

We will store only the first 5 bytes of each given string, even if in
the source code the string is bigger. This is so because unlike in the
original METAFONT, our only use for strings are saying wich Unicode
character each glyph should represent. For this we need only 5 bytes
(at most 4 for the character and a final byte 0). But we could use
them to associate a glyph with ligatures like ``ff'', which also fits
in these 5 bytes.

In the case of symbolic tokens, we need to store their entire name to
know which symbolic token we have. We shoul allocate dynamically
a string to store its name:

\iniciocodigo
@<Local Data Structures (metafont.c)@>+=
#define TYPE_SYMBOLIC 3
struct symbolic_token{
  int type;   // Should be equal 'TYPE_SYMBOLIC'
  void *next;
  char *value;
};
@
\fimcodigo

But this could consume much more memory than needed. Some symbols are
very common and are embedded in the language. These symbols could use
less space. For example, parenthesis, semicolon and commas. We can
represent them with the following structure:

\iniciocodigo
@<Local Data Structures (metafont.c)@>+=
#define TYPE_OPEN_PARENTHESIS  4 // '('
#define TYPE_CLOSE_PARENTHESIS 5 // ')'
#define TYPE_COMMA             6 // ','
#define TYPE_SEMICOLON         7 // ';'
@<METAFONT: Symbolic Token Definition@>
// (...)
struct generic_token{
  int type;   // Should be one of the above
  void *next;
};
@
\fimcodigo

We can define later more reserved symbolic tokens. We just need to
define them to an unique number greater than 6. Any token whose type
is a number greater or equal than 3 is a symbolic token.

Every token have a pointer to a next token. THis happens because
usually they will be part of a linked list. To deallocate the memory
occupied by a token list given a deallocation function and the given
list, we can use the following function:

\iniciocodigo
@<Auxiliary Local Functions (metafont.c)@>=
void free_token_list(void (*dealloc)(void *), void *token_list){
  struct generic_token *p, *p_next;
  if(dealloc == NULL || token_list == NULL)
    return;
  p = token_list;
  while(p != NULL){
    p_next = p -> next;
    if(p -> type == TYPE_SYMBOLIC)
      dealloc(((struct symbolic_token *) p) -> value);
    dealloc(p);
    p = p_next;
  }
}
@
\fimcodigo

Now the funtion that represents our lexer. It will get as argument an
allocation function and a string with a path for the file with
METAFONT source code. It will return a linked list of tokens:

\iniciocodigo
@<Auxiliary Local Functions (metafont.c)@>=
void *lexer(void *(*alloc)(size_t), void (*dealloc)(void *), char *path){
  FILE *fp;
  char c;
  void *first_token = NULL, *last_token = NULL;
  fp = fopen(path, "r");
  if(fp == NULL)
    return NULL;
  while((c = fgetc(fp)) != EOF){
    char next_char = fgetc(fp);
    ungetc(next_char, fp);
    if(c == '\n')
      continue;
    @<Lexer: Rule 1@>
    @<Lexer: Rule 2@>
    @<Lexer: Rule 3@>
    @<Lexer: Rule 4@>
    @<Lexer: Rule 5@>
    @<Lexer: Rule 6@>
    // No rule applied: error
    fprintf(stderr, "ERROR: Unknown character: '%c'\n", c);
    free_token_list(dealloc, first_token);
    return NULL;
  }
  return first_token;
}
@
\fimcodigo

Oh, we used the struct \monoespaco{FILE}, this means that we need to
include the header with data about input/output to be able to read
files:

\iniciocodigo
@<Local Headers (metafont.c)@>=
#include <stdio.h>
@
\fimcodigo

The rules to read tokens consist in reading each line in the source
code applying the following rules for each character in the line:

1) If the next character is a space or a period and is not followed by
a period or a decimal digit, then ignore this character and go to the
next one.

\iniciocodigo
@<Lexer: Rule 1@>=
if(c == ' ' || (c == '.' && next_char != '.' && !isdigit(next_char)))
  continue;
@
\fimcodigo

As we are using \monoespaco{isdigit} function, we need to insert the
following header that declares this function:

\iniciocodigo
@<Local Headers (metafont.c)@>=
#include <ctype.h>
@
\fimcodigo


2) If the character is a percent sign, ignore it and also ignore all
other following characters in this line. Percentage sign is how we
start comments in the language.

\iniciocodigo
@<Lexer: Rule 2@>=
if(c == '%'){
  do{
    c = fgetc(fp);
  } while(c != '\n' && c != EOF);
  continue;
}
@
\fimcodigo

3) If the next character is a decimal digit or a period, then the next
token in numeric. It will be interpreted from the biggest sequence of
decimal digests and a single optional period representing the decimal
dot present in the input.

\iniciocodigo
@<Lexer: Rule 3@>=
if((c == '.' && isdigit(next_char)) || isdigit(c)){
  char buffer[256];
  struct numeric_token *new_token = (struct numeric_token *)
                                      alloc(sizeof(struct numeric_token));
  if(new_token == NULL){
    free_token_list(dealloc, first_token);
    return NULL;
  }
  new_token -> type = TYPE_NUMERIC;
  new_token -> next = NULL;
  int i = 0;
  int number_of_dots = (c == '.');
  buffer[i] = c;
  i ++;
  do{
    c = fgetc(fp);
    if(c == '.')
      number_of_dots ++;
    buffer[i] = c;
    i ++;
  } while(isdigit(c) || (c == '.' && number_of_dots == 1));
  ungetc(c, fp);
  i --;
  buffer[i] = '\0';
  new_token -> value = atof(buffer);
  if(first_token == NULL)
    first_token = last_token = new_token;
  else{
    ((struct generic_token *) last_token) -> next = new_token;
    last_token = new_token;
  }
  continue;
}
@
\fimcodigo

4) If the next character is a double quote, the next token will be a
string. Its content will be all other chacarters until the next double
quote that should be in the same line. If we have a double quote
opening a string, but the same line do not have another double quote
to close the string, this is an error.

\iniciocodigo
@<Lexer: Rule 4@>=
if(c == '"'){
  struct string_token *new_token = (struct string_token *)
                                     alloc(sizeof(struct string_token));
  if(new_token == NULL){
    free_token_list(dealloc, first_token);
    return NULL;
  }
  new_token -> type = TYPE_STRING;
  new_token -> next = NULL;
  int i = 0;
  do{
    c = fgetc(fp);
    if(i < 5){
      new_token -> value[i] = c;
      i ++;
    }
  } while(c != '"' && c != '\n' && c != EOF);
  i --;
  new_token -> value[i] = '\0';
  if(c == '\n' || c == EOF){
    fprintf(stderr, "ERROR: Incomplete String.\n");
    dealloc(new_token);
    free_token_list(dealloc, first_token);
    return NULL;
  }
  if(first_token == NULL)
    first_token = last_token = new_token;
  else{
    ((struct generic_token *) last_token) -> next = new_token;
    last_token = new_token;
  }
  continue;
}
@
\fimcodigo


5) If the next character is a parenthesis, semicolon or a comma, the
next token will be symbolic and composed by that single character.

\iniciocodigo
@<Lexer: Rule 5@>=
if(c == '(' || c == ')' || c == ',' || c == ';'){
  struct generic_token *new_token =
     (struct generic_token *) alloc(sizeof(struct generic_token));
  if(new_token == NULL){
    free_token_list(dealloc, first_token);
    return NULL;
  }
  if(c == '(')
    new_token -> type = TYPE_OPEN_PARENTHESIS;
  else if(c == ')')
    new_token -> type = TYPE_CLOSE_PARENTHESIS;
  else if(c == ';')
    new_token -> type = TYPE_SEMICOLON;
  else
    new_token -> type = TYPE_COMMA;
  new_token -> next = NULL;
  if(first_token == NULL)
    first_token = last_token = new_token;
  else{
    ((struct generic_token *) last_token) -> next = new_token;
    last_token = new_token;
  }
  continue;
}
@
\fimcodigo

6) Otherwise, the next token will be symbolic and will be composed by
the longest sequence of 12 families of characteres:

\iniciocodigo
@<Lexer: Rule 6@>=
{
  char buffer[256];
  int i = 0;
  buffer[0] = '\0';
  // Buffer is read according with subrules 6-a to 6-l
  @<Lexer: Rule A@>
  @<Lexer: Rule B@>
  @<Lexer: Rule C@>
  @<Lexer: Rule D@>
  @<Lexer: Rule E@>
  @<Lexer: Rule F@>
  @<Lexer: Rule G@>
  @<Lexer: Rule H@>
  @<Lexer: Rule I@>
  @<Lexer: Rule J@>
  @<Lexer: Rule K@>
  @<Lexer: Rule L@>
  // Depending on buffer content, generates next token
  @<Lexer: New Reserved Symbolic Token@>
  @<Lexer: New Generic Symbolic Token@>
}
@
\fimcodigo

a) The first family of letters are the uppercase and lowercase
alphabetic letters and underline.

\iniciocodigo
@<Lexer: Rule A@>=
if(isalpha(c) || c == '_'){
  do{
    buffer[i] = c;
    i ++;
    c = fgetc(fp);
  } while(isalpha(c) || c == '_');
  ungetc(c, fp);
  buffer[i] = '\0';
}
@
\fimcodigo

b) The second family is composed by the symbols for greater, equal and
lesser, colon and ``|''.

\iniciocodigo
@<Lexer: Rule B@>=
else if(c == '>' || c == '<' || c == '=' || c == ':' || c == '|'){
  do{
    buffer[i] = c;
    i ++;
    c = fgetc(fp);
  } while(c == '>' || c == '<' || c == '=' || c == ':' || c == '|');
  ungetc(c, fp);
  buffer[i] = '\0';
}
@
\fimcodigo

c) Acute and grave accents.

\iniciocodigo
@<Lexer: Rule C@>=
else if(c == '`' || c == '\''){
  do{
    buffer[i] = c;
    i ++;
    c = fgetc(fp);
  } while(c == '`' || c == '\'');
  ungetc(c, fp);
  buffer[i] = '\0';
}
@
\fimcodigo

d) Plus and minus.

\iniciocodigo
@<Lexer: Rule D@>=
else if(c == '+' || c == '-'){
  do{
    buffer[i] = c;
    i ++;
    c = fgetc(fp);
  } while(c == '+' || c == '-');
  ungetc(c, fp);
  buffer[i] = '\0';
}
@
\fimcodigo

e) Slash, backslash and multiplication symbol.

\iniciocodigo
@<Lexer: Rule E@>=
else if(c == '\\' || c == '/' || c == '*'){
  do{
    buffer[i] = c;
    i ++;
    c = fgetc(fp);
  } while(c == '\\' || c == '/' || c == '*');
  ungetc(c, fp);
  buffer[i] = '\0';
}
@
\fimcodigo

f) Opening brackets

\iniciocodigo
@<Lexer: Rule F@>=
else if(c == '?' || c == '!'){
  do{
    buffer[i] = c;
    i ++;
    c = fgetc(fp);
  } while(c == '?' || c == '!');
  ungetc(c, fp);
  buffer[i] = '\0';
}
@
\fimcodigo

g) Hash sign, ampersand, at sign and dollar sign.

\iniciocodigo
@<Lexer: Rule G@>=
else if(c == '#' || c == '&' || c == '@@' || c == '$'){
  do{
    buffer[i] = c;
    i ++;
    c = fgetc(fp);
  } while(c == '#' || c == '&' || c == '@@' || c == '$');
  ungetc(c, fp);
  buffer[i] = '\0';
}
@
\fimcodigo

h) Circumflex accent and tilde.

\iniciocodigo
@<Lexer: Rule H@>=
else if(c == '^' || c == '~'){
  do{
    buffer[i] = c;
    i ++;
    c = fgetc(fp);
  } while(c == '^' || c == '~');
  ungetc(c, fp);
  buffer[i] = '\0';
}
@
\fimcodigo

i) Opening brackets.

\iniciocodigo
@<Lexer: Rule I@>=
else if(c == '['){
  do{
    buffer[i] = c;
    i ++;
    c = fgetc(fp);
  } while(c == '[');
  ungetc(c, fp);
  buffer[i] = '\0';
}
@
\fimcodigo

j) Closing brackets.

\iniciocodigo
@<Lexer: Rule J@>=
else if(c == ']'){
  do{
    buffer[i] = c;
    i ++;
    c = fgetc(fp);
  } while(c == ']');
  ungetc(c, fp);
  buffer[i] = '\0';
}
@
\fimcodigo

k) Opening and closing braces.

\iniciocodigo
@<Lexer: Rule K@>=
else if(c == '{' || c == '}'){
  do{
    buffer[i] = c;
    i ++;
    c = fgetc(fp);
  } while(c == '{' || c == '}');
  ungetc(c, fp);
  buffer[i] = '\0';
}
@
\fimcodigo

l) Ponto.

\iniciocodigo
@<Lexer: Rule L@>=
else if(c == '.'){
  do{
    buffer[i] = c;
    i ++;
    c = fgetc(fp);
  } while(c == '.');
  ungetc(c, fp);
  buffer[i] = '\0';
}
@
\fimcodigo


After reading the characters for our new symbolic token, we check if
we have a reserved symbolic token: a token representing a language
keyword. We can check this using function \monoespaco{strcmp}
comparing the buffer with some keywords. For now we still will not
show this code because we did not define any reserved symbol, but
anyway we will already include the header to manipulate strings.

\iniciocodigo
@<Local Headers (metafont.c)@>=
#include <string.h>
@
\fimcodigo


If we do not have a reserved symbolic token, and if we have something
in our buffer, then we generate a new generic symbolic token:

\iniciocodigo
@<Lexer: New Generic Symbolic Token@>=
if(buffer[0] != '\0'){
  buffer[255] = '\0';
  size_t buffer_size = strlen(buffer) + 1;
  struct symbolic_token *new_token =
     (struct symbolic_token *) alloc(sizeof(struct symbolic_token));
  if(new_token == NULL){
    free_token_list(dealloc, first_token);
    return NULL;
  }
  new_token -> type = TYPE_SYMBOLIC;
  new_token -> next = NULL;
  new_token -> value = (char *) alloc(buffer_size);
  memcpy(new_token -> value, buffer, buffer_size);
  if(first_token == NULL)
    first_token = last_token = new_token;
  else{
    ((struct generic_token *) last_token) -> next = new_token;
    last_token = new_token;
  }
  continue;
}
@
\fimcodigo

\secao{3. METAFONT Programs}

When we evaluate a METAFONT program, we need two additional
structures. The first one, which we will call \monoespaco{struct
metafont} will contain all the final information extracted from the
typographical font and that will be needed to render each glyph. The
second structure, which we will call \monoespaco{struct context}
represents the current state in our parser and represents information
that we need to know to interpret correctly the token list. This
second structure can be discarded after we read all the tokens from
our font.

The first thing that our parser needs to know is that a METAFONT
program is a list (possibly empty) of statements ended by a symbolic
token \monoespaco{end} or \monoespaco{dump}:

\alinhaverbatim
<Program> -> <List of Statements> end | <List of Statements> dump
\alinhanormal

for us, there will be no difference between these two kind of programs
and symbols \monoespaco{end} and \monoespaco{dump} will be
equivalent. In the original METAFONT, one of them ends programs that
define typographical fonts and the other ends the description of a
base file (something like a basic standard library).

We will define then a new kind of symbolic token to represent this
end-of-file marking:

\iniciocodigo
@<METAFONT: Symbolic Token Definition@>=
#define TYPE_END             8 // The symbolic token 'end' or 'dump'
@
\fimcodigo

And we generate this kind of token when our lexer finds one of these
two keywords (technically METAFONT call them ``sparks'', not
``keywords''):

\iniciocodigo
@<Lexer: New Reserved Symbolic Token@>=
if(!strcmp(buffer, "end") || !strcmp(buffer, "dump")){
  struct generic_token *new_token =
     (struct generic_token *) alloc(sizeof(struct generic_token));
  if(new_token == NULL){
    free_token_list(dealloc, first_token);
    return NULL;
  }
  new_token -> type = TYPE_END;
  new_token -> next = NULL;
  if(first_token == NULL)
    first_token = last_token = new_token;
  else{
    ((struct generic_token *) last_token) -> next = new_token;
    last_token = new_token;
  }
  continue;
}
@
\fimcodigo

The first parser function to be defined is the one that recognizes an
entire program. It basically checks if the program is ended correctly
and if not, generates an error. If the program is ended correctly, it
passes for the next parser function that evaluates list of statements,
marking the beginning and end of such list:

\iniciocodigo
@<Auxiliary Local Functions (metafont.c)@>+=
bool eval_program(struct metafont *mf, struct context *cx,
                  void *token_list){
  struct generic_token *end = (struct generic_token *) token_list;
  //struct generic_token *previous;
  while(end != NULL && end -> type != TYPE_END){
    //previous = end;
    end = (struct generic_token *) (end -> next);
  }
  if(end == NULL){
    fprintf(stderr, "METAFONT: Error: %s: Program not finished with "
                    "'end' or 'dump'.\n", mf -> file);
    return false;
  }
  if(end == token_list)
    return true;
  //return eval_list_of_statements(mf, cx, token_list, previous);
  return true;
}
@
\fimcodigo

The context struct still will not be entirely defined. We will show
its content while each attribute becames necessary:

\iniciocodigo
@<Data Structures (metafont.h)@>=
struct context{
  //@<Attributes (struct context)@>
};
@
\fimcodigo


And as seen above, one of the contents in \monoespaco{struct metafont} is
the file name from where the font was read:

\iniciocodigo
@<Data Structures (metafont.h)@>+=
struct metafont{
  char *file;
  //@<Attributes (struct metafont)@>
};
@
\fimcodigo

\secao{4. List of Statements}

A list of potentially empty statements delimited by a semicolon:

\alinhaverbatim
<List of Statements> -> <Empty> | <Statement> ; <List of Statements>
\alinhanormal

The function that will evaluate and interpret a list of statements is:

\iniciocodigo
@<Local Function Declaration (metafont.c)@>=
bool eval_list_of_statements(struct metafont *mf, struct context *cx,
                            void *begin_token_list, void *end_token_list);
@
\fimcodigo

This function will iterate over each declaration and execute for all
them, in the order thay appear, the function that evaluates individual
statements. For this, the function will find all semicolons and use
them to detect where each statement begins and where they end:

\iniciocodigo
@<Auxiliary Local Functions (metafont.c)@>+=
bool eval_list_of_statements(struct metafont *mf, struct context *cx,
                            void *begin_token_list, void *end_token_list){
  bool ret = true;
  struct generic_token *begin, *end = NULL;
  begin = (struct generic_token *) begin_token_list;
  while(begin != NULL){
    while(begin -> type == TYPE_SEMICOLON){
      if(begin != end_token_list)
        begin = (struct generic_token *) begin -> next;
      else
        begin = NULL;
    }
    end = begin;
    if(end != NULL){
      while(end != end_token_list &&
            ((struct generic_token *) (end -> next)) -> type !=
              TYPE_SEMICOLON)
        end = end -> next;
    }
    if(begin != NULL){
      // eval_statement(mf, cx, begin, end);
      begin = end -> next;
    }
  }
  return ret;
}
@
\fimcodigo

The function above iterates over each statement, ignoring empty
statements. For each non-empty statement, it assures that the
pointers \monoespaco{begin} and \monoespaco{end} delimit the
individual statement. During the iteration we always ensure that we do
not go outside the region delimited by \monoespaco{begin\_token\_list}
and \monoespaco{end\_token\_list}.


%%%%%%%%%%%%%%%%%%%%%%%%%%%%%%%%%%%%%

\secao{References}

\referencia{Knuth, D. E. (1984) ``Literate Programming'', The Computer
  Journal, Volume 27, Issue 2, Pages 97--111.}

\referencia{Knuth, D. E. (1989) ``The METAFONT book'', Addison-Wesley
 Longsman Publishing Co., Inc}


\fim
